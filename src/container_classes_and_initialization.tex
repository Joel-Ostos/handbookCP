\section{Container Classes and Initialization}
\begin{enumerate}
  \item \textbf{Vector:}
    \begin{enumerate}
      \item Class:
	\begin{lstlisting}[language=C++]
	template < class T,
	class Alloc = allocator<T>> class vector;
	\end{lstlisting}
      \item Initialization:
	\begin{enumerate}
	  \item \texttt{vector<int> v}: Declares a vector of integers.
	  \item \texttt{vector<int> v(n)}: Declares a vector of integers of size n.
	  \item \texttt{vector<int> v(n, x)}: Declares a vector of integers of size n, with all elements initialized to x.
	  \item \texttt{vector<int> v = \{1, 2, 3, 4\}}: Declares a vector of integers with the elements 1, 2, 3 and 4.
	  \item \texttt{vector<int> v \{1, 2, 3, 4\}}: Declares a vector of integers with the elements 1, 2, 3 and 4.
	\end{enumerate}
    \end{enumerate}
  \item \textbf{Set:}
    \begin{enumerate}
      \item Class:
	\begin{lstlisting}[language=C++]
	template < class T,
	class Compare = less<T>, \\ View Interesting Classes Chapter for more information
	class Alloc = allocator<T>> class set;
	\end{lstlisting}
      \item Initialization:
	\begin{enumerate}
	  \item \texttt{set<int> s}: Declares a set of integers.
	  \item \texttt{set<int> s \{1, 2, 3, 4\}}: Declares a set of integers with the elements 1, 2, 3 and 4.
	  \item \texttt{set<int> s = \{1, 2, 3, 4\}}: Declares a set of integers with the elements 1, 2, 3 and 4.
	\end{enumerate}
    \end{enumerate}
\end{enumerate}
