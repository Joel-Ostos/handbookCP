\section{Functions}
\section*{Graphs}
\begin{enumerate}
  \item \textbf{DFS:}
	\begin{lstlisting}[language=C++]
	#include <iostream>

	struct Node {
	    int data;
	    Node* left;
	    Node* right;
	    
	    Node(int value) : data(value), left(nullptr), right(nullptr) {}
	};
	\end{lstlisting}
	Preorder:
	\begin{lstlisting}[language=C++]
	void preorder(Node* node) {
		if (node == nullptr) return;
		std::cout << node->data << " ";
		preorder(node->left);
		preorder(node->right);
	}
	\end{lstlisting}
	\begin{lstlisting}[language=C++]
	void inorder(Node* node) {
		if (node == nullptr) return;
		inorder(node->left);
		std::cout << node->data << " ";
		inorder(node->right);
	}
	\end{lstlisting}

	\begin{lstlisting}[language=C++]
	void postorder(Node* node) {
		if (node == nullptr) return;
		postorder(node->left);
		postorder(node->right);
		std::cout << node->data << " ";
	}
	\end{lstlisting}
	Dfs:

	\begin{lstlisting}[language=C++]
		#include <iostream>
#include <vector>

using namespace std;

void DFSUtil(int v, vector<bool> &visited, const vector<vector<int>> &adj) {
    visited[v] = true;
    cout << v << " ";

    for (int i : adj[v]) {
        if (!visited[i]) {
            DFSUtil(i, visited, adj);
        }
    }
}

void DFS(int V, const vector<vector<int>> &adj) {
    vector<bool> visited(V, false);

    for (int i = 0; i < V; ++i) {
        if (!visited[i]) {
            DFSUtil(i, visited, adj);
        }
    }
}

int main() {
    int V = 5;
    vector<vector<int>> adj(V);

    adj[0].push_back(1);
    adj[0].push_back(2);
    adj[1].push_back(3);
    adj[2].push_back(4);

    cout << "DFS starting from vertex 0:\n";
    DFS(V, adj);

    return 0;
}

	\end{lstlisting}
\item BFS:
	\begin{lstlisting}[language=C++]
	#include <iostream>
#include <vector>
#include <queue>

using namespace std;

void BFS(int start, int V, const vector<vector<int>> &adj) {
    vector<bool> visited(V, false);
    queue<int> q;

    visited[start] = true;
    q.push(start);

    while (!q.empty()) {
        int v = q.front();
        q.pop();
        cout << v << " ";

        for (int i : adj[v]) {
            if (!visited[i]) {
                visited[i] = true;
                q.push(i);
            }
        }
    }
}

int main() {
    int V = 5;
    vector<vector<int>> adj(V);

    adj[0].push_back(1);
    adj[0].push_back(2);
    adj[1].push_back(3);
    adj[2].push_back(4);

    cout << "BFS starting from vertex 0:\n";
    BFS(0, V, adj);

    return 0;
}

	\end{lstlisting}
	\item Kruskal:
	\begin{lstlisting}[language=C++]
#include <iostream>
#include <vector>
#include <algorithm>

	using namespace std;

	struct Edge {
		int src, dest, weight;
	};

struct Graph {
	int V, E;
	vector<Edge> edges;
};

struct subset {
	int parent;
	int rank;
};

int find(subset subsets[], int i) {
	if (subsets[i].parent != i) {
		subsets[i].parent = find(subsets, subsets[i].parent);
	}
	return subsets[i].parent;
}

void Union(subset subsets[], int x, int y) {
	int xroot = find(subsets, x);
	int yroot = find(subsets, y);

	if (subsets[xroot].rank < subsets[yroot].rank) {
		subsets[xroot].parent = yroot;
	} else if (subsets[xroot].rank > subsets[yroot].rank) {
		subsets[yroot].parent = xroot;
	} else {
		subsets[yroot].parent = xroot;
		subsets[xroot].rank++;
	}
}

void KruskalMST(Graph& graph) {
	vector<Edge> result;
	int V = graph.V;
	sort(graph.edges.begin(), graph.edges.end(), [](Edge a, Edge b) {
			return a.weight < b.weight;
			});

	subset* subsets = new subset[(V * sizeof(subset))];
	for (int v = 0; v < V; ++v) {
		subsets[v].parent = v;
		subsets[v].rank = 0;
	}

	for (Edge e : graph.edges) {
		int x = find(subsets, e.src);
		int y = find(subsets, e.dest);

		if (x != y) {
			result.push_back(e);
			Union(subsets, x, y);
		}
	}

	cout << "Edges in the MST:\n";
	for (Edge e : result) {
		cout << e.src << " -- " << e.dest << " == " << e.weight << endl;
	}

	delete[] subsets;
}

int main() {
	Graph graph;
	graph.V = 4;
	graph.E = 5;
	graph.edges = {
		{0, 1, 10},
		{0, 2, 6},
		{0, 3, 5},
		{1, 3, 15},
		{2, 3, 4}
	};

	KruskalMST(graph);

	return 0;
}

	\end{lstlisting}
	\item Prims:
	\begin{lstlisting}[language=C++]
	#include <iostream>
#include <vector>
#include <climits>

using namespace std;

int minKey(const vector<int>& key, const vector<bool>& mstSet, int V) {
    int min = INT_MAX, min_index;

    for (int v = 0; v < V; v++)
        if (!mstSet[v] && key[v] < min)
            min = key[v], min_index = v;

    return min_index;
}

void PrimMST(const vector<vector<int>>& graph, int V) {
    vector<int> parent(V);
    vector<int> key(V, INT_MAX);
    vector<bool> mstSet(V, false);

    key[0] = 0;
    parent[0] = -1;

    for (int count = 0; count < V - 1; count++) {
        int u = minKey(key, mstSet, V);
        mstSet[u] = true;

        for (int v = 0; v < V; v++)
            if (graph[u][v] && !mstSet[v] && graph[u][v] < key[v])
                parent[v] = u, key[v] = graph[u][v];
    }

    cout << "Edge \tWeight\n";
    for (int i = 1; i < V; i++)
        cout << parent[i] << " - " << i << " \t" << graph[i][parent[i]] << " \n";
}

int main() {
    int V = 5;
    vector<vector<int>> graph = {
        {0, 2, 0, 6, 0},
        {2, 0, 3, 8, 5},
        {0, 3, 0, 0, 7},
        {6, 8, 0, 0, 9},
        {0, 5, 7, 9, 0}
    };

    PrimMST(graph, V);

    return 0;
}

	\end{lstlisting}
	\item Dijkstra:
	\begin{lstlisting}[language=C++]
	#include <iostream>
#include <vector>
#include <queue>
#include <climits>

using namespace std;

void dijkstra(const vector<vector<pair<int, int>>> &graph, int src) {
    int V = graph.size();
    vector<int> dist(V, INT_MAX);
    priority_queue<pair<int, int>, vector<pair<int, int>>, greater<pair<int, int>>> pq;

    pq.push({0, src});
    dist[src] = 0;

    while (!pq.empty()) {
        int u = pq.top().second;
        pq.pop();

        for (auto &neighbor : graph[u]) {
            int v = neighbor.first;
            int weight = neighbor.second;

            if (dist[u] + weight < dist[v]) {
                dist[v] = dist[u] + weight;
                pq.push({dist[v], v});
            }
        }
    }

    cout << "Vertex \t Distance from Source\n";
    for (int i = 0; i < V; ++i)
        cout << i << " \t " << dist[i] << "\n";
}

int main() {
    int V = 5;
    vector<vector<pair<int, int>>> graph(V);

    graph[0].push_back({1, 10});
    graph[0].push_back({4, 5});
    graph[1].push_back({2, 1});
    graph[1].push_back({4, 2});
    graph[2].push_back({3, 4});
    graph[3].push_back({0, 7});
    graph[3].push_back({2, 6});
    graph[4].push_back({1, 3});
    graph[4].push_back({2, 9});
    graph[4].push_back({3, 2});

    dijkstra(graph, 0);

    return 0;
}

	\end{lstlisting}
	\item Bellman:
	\begin{lstlisting}[language=C++]
	#include <iostream>
#include <vector>
#include <climits>

using namespace std;

struct Edge {
    int src, dest, weight;
};

void bellmanFord(const vector<Edge> &edges, int V, int src) {
    vector<int> dist(V, INT_MAX);
    dist[src] = 0;

    for (int i = 1; i <= V - 1; ++i) {
        for (const auto &edge : edges) {
            if (dist[edge.src] != INT_MAX && dist[edge.src] + edge.weight < dist[edge.dest]) {
                dist[edge.dest] = dist[edge.src] + edge.weight;
            }
        }
    }

    for (const auto &edge : edges) {
        if (dist[edge.src] != INT_MAX && dist[edge.src] + edge.weight < dist[edge.dest]) {
            cout << "Graph contains negative weight cycle\n";
            return;
        }
    }

    cout << "Vertex \t Distance from Source\n";
    for (int i = 0; i < V; ++i)
        cout << i << " \t " << dist[i] << "\n";
}

int main() {
    int V = 5;
    vector<Edge> edges = {
        {0, 1, -1},
        {0, 2, 4},
        {1, 2, 3},
        {1, 3, 2},
        {1, 4, 2},
        {3, 2, 5},
        {3, 1, 1},
        {4, 3, -3}
    };

    bellmanFord(edges, V, 0);

    return 0;
}
	\end{lstlisting}
\end{enumerate}
